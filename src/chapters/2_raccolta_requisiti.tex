\section{Funzionalit\`a offerte}
L'applicazione prevede la presenza di un Wallet associato a ciascun tenant, che pu\`o essere ricaricato tramite pagamenti effettuati con \textbf{Stripe}.
All'interno del sistema sono previsti dei \textbf{workflow}, processi che utilizzano dati provenienti da banche dati esterne per effettuare analisi di vario tipo.
Ogni operazione all'interno del workflow ha un costo, e per ognuna di queste viene scalato un determinato importo in base a un listino prezzi associato al tenant di appartenenza dell'utente.
Quando un'operazione viene inserita nella coda pagamenti, questa viene registrata nel database e il relativo importo viene scalato automaticamente dal Wallet del tenant.
\\\\
In questo capitolo ci concentreremo sulle interfacce del Frontend. Per quanto riguarda gli altri requisiti funzionali, verranno discussi in dettaglio nei capitoli successivi.
L'interfaccia prevede la suddivisione in due schermate principali, una per la gestione del Wallet e una per la gestione dei listini prezzi.

\section{Sezione per la gestione del Wallet}
Per monitorare il Wallet del suo tenant di appartenenza, l'utente ha a disposizione una dashboard in cui pu\`o visualizzare lo storico delle transazioni e ricaricare il Wallet.

\begin{figure}[H]
  \centering
  \includegraphics[width=13cm]{images/gestione-wallet/mock-gestione-wallet.png}
  \caption{Wireframe della schermata di Gestione Wallet}
\end{figure}

Nella parte superiore sono presenti dei widget che riportano alcune informazioni in modo da essere facilmente accessibili:
\begin{enumerate}
  \item Credito disponibile
  \item Credito speso  nel mese corrente
  \item Credito depositato nel mese corrente
  \item Differenza tra credito speso e credito depositato nel mese corrente
\end{enumerate}
Gli ultimi tre grafici riportano anche l'andamento rispetto al mese precedente.
\\\\
Il bottone \textbf{Ricarica} apre una schermata che permette di caricare un importo arbitrario, con la possibilit\`a di scegliere
dei tagli prestabiliti.

\begin{figure}[H]
  \centering
  \includegraphics[width=8.5cm]{images/gestione-wallet/mock-seleziona-importo.png}
  \caption{Wireframe della schermata di selezione importo }
\end{figure}
Procedendo al pagamento si conferma l'importo, passando quindi alla schermata di pagamento di Stripe.

\begin{figure}[H]
  \centering
  \includegraphics[width=5.5cm]{images/gestione-wallet/mock-stripe.png}
  \caption{Wireframe della schermata di pagamento con Stripe }
\end{figure}

Il bottone \textbf{Filtri} apre una schermata laterale dove \`e possibile selezionare le propriet\`a per cui deve essere filtrata la lista delle transazioni.

\begin{figure}[H]
  \centering
  \includegraphics[width=6cm]{images/gestione-wallet/mock-filtri-ricerca.png}
  \caption{Wireframe della schermata dei filtri}
\end{figure}

\section{Sezione per la gestione dei listini prezzi}
L'amministratore ha a disposizione una schermata in cui pu\`o aggiungere, modificare o cancellare un listino prezzi.
\begin{figure}[H]
  \centering
  \includegraphics[width=13cm]{images/gestione-listini/listini-prezzi-list.png}
  \caption{Wireframe della schermata con la lista dei listini prezzi}
\end{figure}
Il bottone \textbf{Aggiungi Listino} reindirizza ad una pagina che consente di inserire i dati per creare un nuovo listino prezzi, come nella Figura \ref{aggiungilistino}.
\\\\
Nella tabella invece sono presenti 3 azioni per ogni listino prezzi.
La prima apre il dettaglio indicato nella Figura \ref{dettagliolistino}.
La seconda reindirizza ad una pagina analoga a quella della Figura \ref{aggiungilistino}, che permette di modificare il listino prezzi.
La terza apre una schermata che chiede se si vuole veramente procedere all'eliminazione, che se confermata procede all'eliminazione.
\begin{figure}[H]
  \centering
  \includegraphics[width=13cm]{images/gestione-listini/dettaglio-listino.png}
  \caption{Wireframe della schermata del dettaglio di un listino prezzi}
  \label{dettagliolistino}
\end{figure}

Nel dettaglio del listino prezzi sono presenti due sezioni: nella sezione \textbf{Informazioni Base} sono presenti la descrizione e la fascia di prezzo, mentre nella sezione
\textbf{Prezzi} \`e presente una tabella per ogni banca dati con il relativo costo di ogni operazione all'interno del listino prezzi. Ogni tabella \`e chiudibile.
Con il bottone di edit (l'icona a forma di matita) si viene reindirizzati ad una pagina che permette di modificare il listino, analoga alla schermata presente nella Figura \ref{aggiungilistino}.

\begin{figure}[H]
  \centering
  \includegraphics[width=13cm]{images/gestione-listini/add-listino.png}
  \caption{Wireframe della schermata di creazione listino prezzi}
  \label{aggiungilistino}
\end{figure}
Nella sezione \textbf{Informazioni Base} \`e possibile inserire una descrizione e una fascia di prezzo.
Nella sezione \textbf{Prezzi} \`e presente una tabella per ogni banca dati che riporta i costi inseriti. Anche qui, ogni tabella \`e chiudibile.
Cliccare su una riga apre una schermata analoga a quella indicata nella Figura \ref{aggiungiprezzo}, con la differenza che la selezione del tipo operazione
non \`e presente in quanto il campo \`e preselezionato. Questa schermata consente quindi di modificare il prezzo inserito in precedenza,

Per terminare la creazione di un Listino \`e necessario inserire un importo per tutte le operazioni di pagamento.
\\\\
Il bottone \textbf{Aggiungi Prezzo} apre la schermata di aggiunta del prezzo in Figura \ref{aggiungiprezzo}
\begin{figure}[H]
  \centering
  \includegraphics[width=8.5cm]{images/gestione-listini/aggiungi-prezzo.png}
  \caption{Wireframe della schermata per inserire un importo}
  \label{aggiungiprezzo}
\end{figure}

Da questa schermata \`e possibile selezionare un tipo operazione e specificare il suo importo, avendo sempre sotto controllo il suo costo reale di interrogare la banca dati esterna.
