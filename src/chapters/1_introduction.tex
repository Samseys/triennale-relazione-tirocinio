Questa relazione presenta il lavoro svolto durante il tirocinio presso l'azienda \textit{RESI Informatica}, con l'obiettivo di sviluppare un modulo per la gestione
dei conti e dei relativi listini prezzi per un'applicazione multi-tenant. Il modulo \`e composto da una parte Back-end scritta in \textit{Java} con \textit{SpringBoot}, e da una parte Front-end realizzata con \textit{Angular}.

L'applicazione prevede un Wallet associato a ogni tenant che pu\`o essere ricaricato pagando con \textit{Stripe}.
Il back-end resta in ascolto su una coda del message broker \textit{RabbitMQ}, sulla quale vengono inviati messaggi relativi ai costi sostenuti durante l'esecuzione delle operazioni all'interno di un workflow.
Per ognuna di queste operazioni viene scalato un determinato importo, che dipende dal listino prezzi associato al tenant.
Come DMBS \`e stato scelto \textit{PostgreSQL} per la sua affidabilit\`a e scalabilit\`a.

Dal lato Front-end, \`e prevista una pagina per la gestione del Wallet dove \`e possibile effettuare ricariche e visualizzare lo storico delle operazioni.
\`e inoltre presente una sezione per la creazione e modifica dei listini prezzi.
L'applicazione viene containerizzata con \textit{Docker} ed eseguita, insieme agli altri servizi dell'applicazione, utilizzando \textit{Kubernetes}.
