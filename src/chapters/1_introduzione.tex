Questa relazione presenta il lavoro svolto durante il tirocinio presso l'azienda \textit{RESI Informatica}, che ha avuto come obiettivo la creazione di un modulo per la
gestione di conti virtuali e dei relativi listini prezzi per un'applicazione multi-tenant con architettura Event-Driven.

L'applicazione prevede la presenza di un Wallet associato a ciascun tenant, che pu\`o essere ricaricato tramite pagamenti effettuati con \textit{Stripe}.
All'interno del sistema sono previsti dei \textit{workflow}, processi che utilizzano dati provenienti da banche dati esterne per effettuare analisi di vario tipo.
Ogni operazione all'interno del workflow ha un costo, e per ognuna di queste viene scalato un determinato importo in base a un listino prezzi associato al tenant di appartenenza dell'utente.
Quando un'operazione viene inserita nella coda pagamenti, questa viene registrata nel database e il relativo importo \`e scalato automaticamente dal Wallet del tenant.

Per la gestione del Wallet \`e presente un'interfaccia che consente di ricaricare il saldo, monitorare lo storico delle transazioni e di visualizzare il saldo disponibile.
\`E presente anche una sezione per la configurazione dei listini prezzi, dove gli amministratori possono creare e modificare i prezzi delle singole operazioni.
