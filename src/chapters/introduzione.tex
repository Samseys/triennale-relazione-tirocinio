\chapter{Introduzione}

\section{Scopo del tirocinio}
Questa relazione presenta il lavoro svolto durante il tirocinio presso l'azienda \textbf{RESI Informatica}, che ha avuto come obiettivo la creazione di un servizio per la
gestione di conti virtuali e dei relativi listini prezzi per un'applicazione multi-tenant con architettura event-driven e microservizi.

Il nome deciso per questo servizio \`e \textbf{Credit Manager}.

\section{Descrizione di Sentinel}
Sentinel \`e un applicativo progettato per automatizzare l'analisi di portafogli di crediti deteriorati (NPL, Non-Performing Loans), fornendo un supporto per le societ\`a
di recupero crediti nella valutazione dei possibili rischi.

Tra le funzionalit\`a chiave offerte da Sentinel troviamo:
\begin{itemize}
  \item \textbf{Ricerca e aggregazione} - Il sistema ha la capacit\`a di aggregare grandi quantit\`a di dati estrapolati da banche dati esterne quali SISTER (Agenzia delle entrate), Cerved e altre, e di utilizzarli
    per effettuare un'analisi approfondita del portafoglio elaborato.
  \item \textbf{Calcolo del potenziale recupero} - Tra i dati che Sentinel recupera abbiamo dati relativi al valore degli asset in possesso dei soggetti debitori, che possono
    quindi essere utilizzati per effettuare una stima dei ricavi.
  \item \textbf{Supporto decisionale per le strategie di recupero} - Attraverso le analisi che Sentinel effettua sul portafoglio, \`e in grado di suggerire delle strategie
    da seguire per il recupero in modo da minimizzare i rischi.
  \item \textbf{Generazione automatica di documenti} - Sentinel \`e in grado di generare i documenti necessari per intraprendere azioni legali.
  \item \textbf{Interfaccia semplice da utilizzare} - Sentinel \`e dotato di molteplici dashboard che consentono di visualizzare in maniera comprensiva lo stato di un portafoglio,
    le pratiche di credito collegate, e di effettuare simulazioni.
\end{itemize}

\section{Organizzazione e conduzione del lavoro}
Le fasi del progetto sono state scandite dai seguenti punti:
\begin{itemize}
  \item Scelta della tech stack
  \item Apprendimento delle basi di Angular, Spring e RabbitMQ
  \item Progettazione della base dati
  \item Progettazione delle interfacce del frontend
  \item Realizzazione del backend
  \item Realizzazione del frontend con adattamento del backend per i nuovi sviluppi
\end{itemize}
Durante il tirocinio ho lavorato a stretto contatto con \textbf{Luca Pittori} e \textbf{Roberto Tosolini}, con la supervisione del mio relatore aziendale \textbf{Mauro Felici}.
Con il primo ho affrontato la parte di progettazione, scelta della tech stack, apprendimento iniziale delle tecnologie. Con il secondo ho affrontato la parte di realizzazione del backend,
di progettazione dell'interfaccia e di realizzazione del frontend.

Come piattaforma per hostare il codice sorgente \`e stata utilizzata un'istanza locale di GitLab, mentre per la comunicazione sono stati utilizzati Skype e Microsoft Teams.
Il progetto \`e stato realizzato seguendo la metologia Agile, i cui principi sono riassunti nel seguente manifesto\cite{beck2001manifesto}:

\textit{\textbf{Gli individui e le interazioni} pi\`u che i processi e gli strumenti\\
  \textbf{Il software funzionante} pi\`u che la documentazione esaustiva\\
  \textbf{La collaborazione col cliente} pi\`u che la negoziazione dei contratti\\
  \textbf{Rispondere al cambiamento} pi\`u che seguire un piano\\\\
  Ovvero, fermo restando il valore delle voci a destra,\\
  consideriamo pi\`u importanti le voci a sinistra.
}

In particolare, ci sono state continue interazioni per raffinare gli aspetti progettuali e implementativi, permettendo cicli di sviluppo pi\`u brevi alla fine dei quali
sono stati raggiunti dei risultati concreti per il progetto.
