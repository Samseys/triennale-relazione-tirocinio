\chapter{Introduzione}

\section{Scopo del tirocinio}
Questa relazione presenta il lavoro svolto durante il tirocinio presso l'azienda \textbf{RESI Informatica}, che ha avuto come obiettivo la creazione di un servizio per la
gestione di conti virtuali e dei relativi listini prezzi per un'applicazione multi-tenant con architettura event-driven e microservizi.
\\
Il nome deciso per questo servizio \`e \textbf{Credit Manager}.

\section{Descrizione di Sentinel}
Sentinel \`e un'applicativo progettato per automatizzare l'analisi di portafogli di crediti deteriorati (NPL, Non-Performing Loans), fornendo un supporto per le societ\`a
di recupero crediti nella valutazione dei possibili rischi.
\\
Tra le funzionalit\`a chiave offerte da Sentinel troviamo:
\begin{itemize}
  \item \textbf{Ricerca e aggregazione} - Il sistema ha la capacit\`a di aggregare grandi quantit\`a di dati estrapolati da banche dati esterne quali SISTER (Agenzia delle entrate), Cerved e altre, e di utilizzarli
    per effettuare un'analisi approfondita del portafoglio elaborato.
  \item \textbf{Calcolo del potenziale recupero} - Tra i dati che Sentinel recupera abbiamo dati relativi al valore degli asset in possesso dei soggetti debitori, che possono
    quindi essere utilizzati per effettuare una stima dei ricavi.
  \item \textbf{Supporto decisionale per le strategie di recupero} - Attraverso le analisi che Sentinel effettua sul portafoglio, \`e in grado di suggerire delle strategie
    da seguire per il recupero per minimizzare i rischi.
  \item \textbf{Generazione automatica di documenti} - Sentinel \`e in grado di generare i documenti necessari per intraprendere azioni legali.
  \item \textbf{Interfaccia semplice da utilizzare} - Sentinel \`e dotato di molteplici dashboard che consentono di visualizzare in maniera comprensiva lo stato di un portafoglio,
    le pratiche di credito collegate, e di effettuare simulazioni.
\end{itemize}

\section{Organizzazione e conduzione del lavoro}
